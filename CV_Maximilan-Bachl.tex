%% start of file `template.tex'.
%% Copyright 2006-2012 Xavier Danaux (xdanaux@gmail.com).
%
% This work may be distributed and/or modified under the
% conditions of the LaTeX Project Public License version 1.3c,
% available at http://www.latex-project.org/lppl/.


\documentclass[11pt,a4paper]{moderncv}   % possible options include font size ('10pt', '11pt' and '12pt'), paper size ('a4paper', 'letterpaper', 'a5paper', 'legalpaper', 'executivepaper' and 'landscape') and font family ('sans' and 'roman')

% moderncv themes
\moderncvstyle{casual}                        % style options are 'casual' (default), 'classic', 'oldstyle' and 'banking'
\moderncvcolor{blue}                          % color options 'blue' (default), 'orange', 'green', 'red', 'purple', 'grey' and 'black'
%\renewcommand{\familydefault}{\sfdefault}    % to set the default font; use '\sfdefault' for the default sans serif font, '\rmdefault' for the default roman one, or any tex font name
%\nopagenumbers{}                             % uncomment to suppress automatic page numbering for CVs longer than one page

% character encoding
%\usepackage[utf8]{inputenc}                  % if you are not using xelatex ou lualatex, replace by the encoding you are using
%\usepackage{CJKutf8}                         % if you need to use CJK to typeset your resume in Chinese, Japanese or Korean

% adjust the page margins
\usepackage[scale=0.75]{geometry}
%\setlength{\hintscolumnwidth}{3cm}           % if you want to change the width of the column with the dates
%\setlength{\makecvtitlenamewidth}{10cm}      % for the 'classic' style, if you want to force the width allocated to your name and avoid line breaks. be careful though, the length is normally calculated to avoid any overlap with your personal info; use this at your own typographical risks...

\usepackage{graphicx}
\usepackage{amssymb}
\usepackage{polyglossia}
\setdefaultlanguage{english}

\usepackage{fontspec,xltxtra,xunicode}
\defaultfontfeatures{Mapping=tex-text}
%\setromanfont[Mapping=tex-text]{Hoefler Text}
%\setsansfont[Scale=MatchLowercase,Mapping=tex-text]{Ubuntu}
%\setmonofont[Scale=MatchLowercase]{Menlo}
\setmainfont[Scale=MatchLowercase,Mapping=tex-text]{Ubuntu}

% personal data
\firstname{Maximilian}
\familyname{Bachl}
\title{curriculum vitæ}					% optional, remove the line if not wanted
\address{37 Boulevard Jourdan}{75014 Paris, France}    % optional, remove the line if not wanted
\mobile{+33~7~82~30~20~45}                     % optional, remove the line if not wanted
% \phone{+2~(345)~678~901}                      % optional, remove the line if not wanted
% \fax{+3~(456)~789~012}                        % optional, remove the line if not wanted
\email{maximilian.bachl@gmail.com}                          % optional, remove the line if not wanted
% \homepage{www.johndoe.com}                    % optional, remove the line if not wanted
% \extrainfo{additional information}            % optional, remove the line if not wanted
\photo[64pt][0.4pt]{picture}                  % '64pt' is the height the picture must be resized to, 0.4pt is the thickness of the frame around it (put it to 0pt for no frame) and 'picture' is the name of the picture file; optional, remove the line if not wanted
% \quote{Some quote (optional)}                 % optional, remove the line if not wanted

% to show numerical labels in the bibliography (default is to show no labels); only useful if you make citations in your resume
%\makeatletter
%\renewcommand*{\bibliographyitemlabel}{\@biblabel{\arabic{enumiv}}}
%\makeatother

% bibliography with mutiple entries
%\usepackage{multibib}
%\newcites{book,misc}{{Books},{Others}}
%----------------------------------------------------------------------------------
%            content
%----------------------------------------------------------------------------------
\begin{document}
%-----       letter       ---------------------------------------------------------
% recipient data
\recipient{\employer{}}{}
%\date{January 01, 1984}
\opening{Dear Mr Benbadis,}
\closing{Sincerely yours,\\
\vspace*{0.4cm}
\includegraphics{unterschrift.png}
\vspace*{-1.1cm}}
\enclosure[Attachment]{curriculum vitæ}     % use an optional argument to use a string other than "Enclosure", or redefine \enclname
\newcommand{\employer}{Thales advanced studies TAI Lab}
\makelettertitle

I was looking for a challenging and fascinating internship where I can use my acquired skills practically. Because of my personal interest and experience in telecommunication technology and networking I believe that the position at \employer{} is an excellent opportunity for me and that I can contribute a lot and thus I am applying at your institution.

During my Bachelor studies and my student job in the field of LTE research I became fascinated by networking technology and I developed a keen interest in it. In the course of my Master studies in the field of Internet Technology I gained deep knowledge about the concepts of Software Defined Networking and got intrigued by this novel approach which will lead to groundbreaking changes in the future of telecommunication. Besides my studies I am also interested in traveling which was the reason for me to join an international student association where I was a board member and organized international gatherings. Furthermore I decided to pursue an international Master program combining technology and innovation and I believe that this has enabled me to be open to new ideas and understand people from different cultures in a better way which is a useful skill in an international environment like at Thales. 

As an innovation-minded person and problem solver I believe that my skills and experience greatly match the offered position at Thales. I am looking forward to your response and it would be a great pleasure for me to meet you in person. Thank you for your time and consideration.

\makeletterclosing
%
%\clearpage

%-----       resume       ---------------------------------------------------------
%\makecvtitle
%
%\section{General information}
%\cvitem{Day of birth}{26.6.1992}  % arguments 3 to 6 can be left empty
%\cvitem{Nationality}{Austrian}  % arguments 3 to 6 can be left empty
%\cvitem{Place of birth}{Vienna}  % arguments 3 to 6 can be left empty
%
%\section{Education}
%\cventry{From 9.2015}{EIT Digital, second year}{Université Pierre et Marie Curie}{Paris}{}{}
%\cventry{9.2014--6.2015}{EIT Digital, first year}{University of Trento}{Trento, Italy}{}{The specialization of the study program is \textit{Internet Technology and Architecture} \newline
%Average grade 28.5 out of 30 points}
%\cventry{10.2011--8.2014}{Bachelor of Computer Science}{Technische Universität Berlin}{Berlin}{}{Average grade of 1.8 where 1 is the best and 5 the worst}  % arguments 3 to 6 can be left empty
%\cventry{2002--2010}{Gymnasium}{GRG17 Geblergasse}{Vienna}{}{Graduated in June 2010 with an average grade of 1.25 where 1 is the best and 5 the worst}
%
%\section{Bachelor thesis}
%\cvitem{Title}{\textbf{Impact of Packet-Loss on Localization and Subjective Quality in 3D-Telephone calls}}
%\cvitem{Professor}{Prof.~Dr.-Ing.~Sebastian Möller}
%\cvitem{Supervisors}{Janto Skowronek, Dennis Guse}
%\cvitem{Short Abstract}{The impact of packet-loss on the quality of a 3D audio reproduction system in the case of multi party telephone conferences is investigated. I conducted a user study with 28 participants. The evaluation was carried out using MATPLOTLIB and Gnu R.}
%
%%\section{Master thesis}
%%\cvitem{title}{\emph{Title}}
%%\cvitem{supervisors}{Supervisors}
%%\cvitem{description}{Short thesis abstract}
%
%\section{Experience}
%\subsection{Work experience}
%\cventry{5.2013--5.2014}{Software Developer}{Fraunhofer Heinrich Hertz Institute}{Berlin}{}{I worked in the department for Wireless Networks where I participated in the development of a software defined radio implementation of LTE.
%My major tasks were to implement the demapper, descrambler and demodulator according to the LTE specifications and an associated graphical front-end. \\
%I accomplished these tasks mainly using MATLAB and C.}
%\cventry{since 11.2012}{Self Employed}{}{}{}{I worked for start-ups, doing mainly frontend development with Ruby on Rails and recent HTML5 technologies. Also I developed websites for private persons and small companies.}
%\cventry{9.2012}{Internship}{ATOS}{Vienna}{}{Development of various commandline utilities and graphical Programs for Windows, e.g.\ a graphical Task Manager written in Visual Basic .NET.}
%\cventry{8.2011}{Internship}{OMV}{Vienna}{}{Administration of databases and translation of technical manuals from German to English.}
%\cventry{8.2009}{Internship}{OMV}{Vienna}{}{Development of an automated email software using Python.}
%%\newline{}
%%Detailed achievements:
%%\begin{itemize}
%%\item Achievement 1;
%%\item Achievement 2, with sub-achievements:
%%  \begin{itemize}
%%  \item Sub-achievement (a);
%%  \item Sub-achievement (b), with sub-sub-achievements (don't do this!);
%%    \begin{itemize}
%%    \item Sub-sub-achievement i;
%%    \item Sub-sub-achievement ii;
%%    \item Sub-sub-achievement iii;
%%    \end{itemize}
%%  \item Sub-achievement (c);
%%  \end{itemize}
%%\item Achievement 3.
%%\end{itemize}}
%%\cventry{year--year}{Job title}{Employer}{City}{}{Description line 1\newline{}Description line 2}
%\subsection{Miscellaneous}
%\cventry{2013--2014}{Board Member}{AEGEE-Berlin e.V.}{Berlin}{}{AEGEE is the largest transnational, interdisciplinary student organization in Europe. It promotes an equal, democratic and unified Europe, open to all across national borders. I was in the board of the association in Berlin as the treasurer and organized several international events with up to 30 participants. These gatherings had a focus on cultural exchange and learning about other cultures in Europe.}
%\cventry{2010--2011}{Civil Service}{Jewish Community of Vienna}{Vienna}{}{Support of holocaust survivors and scientific research activity in the archives}
%
%\section{Languages}
%%\cvitemwithcomment{Deutsch}{Muttersprache}{}
%%\cvitemwithcomment{Englisch}{Ausgezeichnet}{}
%\cvitem{German}{First language}
%\cvitem{English}{C2 Level (TOEFL test 115/120 points)}
%\cvitem{Italian}{Intermediate}
%\cvitem{French}{Basics}
%%\cvitemwithcomment{Language 3}{Skill level}{Comment}
%
%\section{Programming abilities}
%%\cvdoubleitem{category 1}{XXX, YYY, ZZZ}{category 4}{XXX, YYY, ZZZ}
%%\cvdoubleitem{category 2}{XXX, YYY, ZZZ}{category 5}{XXX, YYY, ZZZ}
%%\cvdoubleitem{category 3}{XXX, YYY, ZZZ}{category 6}{XXX, YYY, ZZZ}
%\cvitem{OS}{Windows, Linux, Mac OS X and Unix in general}
%\cvitem{Languages}{MATLAB, Gnu R, C, Ruby, Python, Java, JavaScript, Assembly, Bash}
%\cvitem{Web}{Experience with Ruby on Rails, Django and HTML5 and AJAX, CSS3 and CSS preprocessors. Furthermore I have experience with commonly used programming interfaces like the Google Maps API.}
%\cvitem{Miscellaneous}{\LaTeX, MS Office, version control software like git and databases like PostgreSQL and MongoDB.\newline Experience with mobile communications (GSM, UMTS and LTE). I learned about mobile communications at Fraunhofer HHI where I worked on the LTE specification and because of my Bachelor's thesis which I wrote at Deutsche Telekom. 
%%\newline In this thesis I investigate the influence of packet-loss on the quality of 3D-audio-conferences. It included a study with over 24 people in which the subjects had to rate presented audio samples.
%}
%
%\section{Hobbies}
%\cvitem{Sport}{Jogging, cycling, skiing, strength training}
%\cvitem{Interests}{Programming, politics, philosophy, gardening, traveling}
%
%%\section{Extra 1}
%%\cvlistitem{Item 1}
%%\cvlistitem{Item 2}
%%\cvlistitem{Item 3}
%%
%%\renewcommand{\listitemsymbol}{-~}            % change the symbol for lists
%%
%%\section{Extra 2}
%%\cvlistdoubleitem{Item 1}{Item 4}
%%\cvlistdoubleitem{Item 2}{Item 5\cite{book1}}
%%\cvlistdoubleitem{Item 3}{}
%
%% Publications from a BibTeX file without multibib
%%  for numerical labels: \renewcommand{\bibliographyitemlabel}{\@biblabel{\arabic{enumiv}}}
%%  to redefine the heading string ("Publications"): \renewcommand{\refname}{Articles}
%\nocite{*}
%\bibliographystyle{plain}
%\bibliography{publications}                   % 'publications' is the name of a BibTeX file
%
%% Publications from a BibTeX file using the multibib package
%%\section{Publications}
%%\nocitebook{book1,book2}
%%\bibliographystylebook{plain}
%%\bibliographybook{publications}              % 'publications' is the name of a BibTeX file
%%\nocitemisc{misc1,misc2,misc3}
%%\bibliographystylemisc{plain}
%%\bibliographymisc{publications}              % 'publications' is the name of a BibTeX file

\end{document}


%% end of file `template.tex'.
